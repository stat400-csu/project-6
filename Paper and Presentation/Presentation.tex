% Options for packages loaded elsewhere
\PassOptionsToPackage{unicode}{hyperref}
\PassOptionsToPackage{hyphens}{url}
\documentclass[
  ignorenonframetext,
]{beamer}
\newif\ifbibliography
\usepackage{pgfpages}
\setbeamertemplate{caption}[numbered]
\setbeamertemplate{caption label separator}{: }
\setbeamercolor{caption name}{fg=normal text.fg}
\beamertemplatenavigationsymbolsempty
% remove section numbering
\setbeamertemplate{part page}{
  \centering
  \begin{beamercolorbox}[sep=16pt,center]{part title}
    \usebeamerfont{part title}\insertpart\par
  \end{beamercolorbox}
}
\setbeamertemplate{section page}{
  \centering
  \begin{beamercolorbox}[sep=12pt,center]{section title}
    \usebeamerfont{section title}\insertsection\par
  \end{beamercolorbox}
}
\setbeamertemplate{subsection page}{
  \centering
  \begin{beamercolorbox}[sep=8pt,center]{subsection title}
    \usebeamerfont{subsection title}\insertsubsection\par
  \end{beamercolorbox}
}
% Prevent slide breaks in the middle of a paragraph
\widowpenalties 1 10000
\raggedbottom
\AtBeginPart{
  \frame{\partpage}
}
\AtBeginSection{
  \ifbibliography
  \else
    \frame{\sectionpage}
  \fi
}
\AtBeginSubsection{
  \frame{\subsectionpage}
}
\usepackage{iftex}
\ifPDFTeX
  \usepackage[T1]{fontenc}
  \usepackage[utf8]{inputenc}
  \usepackage{textcomp} % provide euro and other symbols
\else % if luatex or xetex
  \usepackage{unicode-math} % this also loads fontspec
  \defaultfontfeatures{Scale=MatchLowercase}
  \defaultfontfeatures[\rmfamily]{Ligatures=TeX,Scale=1}
\fi
\usepackage{lmodern}
\ifPDFTeX\else
  % xetex/luatex font selection
\fi
% Use upquote if available, for straight quotes in verbatim environments
\IfFileExists{upquote.sty}{\usepackage{upquote}}{}
\IfFileExists{microtype.sty}{% use microtype if available
  \usepackage[]{microtype}
  \UseMicrotypeSet[protrusion]{basicmath} % disable protrusion for tt fonts
}{}
\makeatletter
\@ifundefined{KOMAClassName}{% if non-KOMA class
  \IfFileExists{parskip.sty}{%
    \usepackage{parskip}
  }{% else
    \setlength{\parindent}{0pt}
    \setlength{\parskip}{6pt plus 2pt minus 1pt}}
}{% if KOMA class
  \KOMAoptions{parskip=half}}
\makeatother
\setlength{\emergencystretch}{3em} % prevent overfull lines
\providecommand{\tightlist}{%
  \setlength{\itemsep}{0pt}\setlength{\parskip}{0pt}}
\usepackage{bookmark}
\IfFileExists{xurl.sty}{\usepackage{xurl}}{} % add URL line breaks if available
\urlstyle{same}
\hypersetup{
  pdftitle={Sequential Experimental Design for Predator-Prey Models},
  pdfauthor={Juliette Dashe, Jake Brisnehan, Mia Krause},
  hidelinks,
  pdfcreator={LaTeX via pandoc}}

\title{Sequential Experimental Design for Predator-Prey Models}
\subtitle{Stat 400 Final Project}
\author{Juliette Dashe, Jake Brisnehan, Mia Krause}
\date{2025-12-12}

\begin{document}
\frame{\titlepage}

\begin{frame}{Motivation: The Ecological Problem}
\phantomsection\label{motivation-the-ecological-problem}
\textbf{How do predators respond to prey density?}

\begin{itemize}
\tightlist
\item
  {\textbf{Holling's Type II (Glutton):}} Eats fast, gets full quickly.
  \[
  \frac{dN}{dt} =  -\frac{aN}{1+aT_{h}N}
  \]
\item
  {\textbf{Holling's Type III (Learner):}} Slow start (learning),
  accelerates, then gets full. \[
  \frac{dN}{dt} = -\frac{aN^2}{1+aT_{h}N^2}
  \]
\end{itemize}

where N = prey density, a = attack rate, \(T_{h}\) = handling time

\begin{frame}{Visualizing the Challenge}
\phantomsection\label{visualizing-the-challenge}
\textbf{The Issue:} These biological behaviors look mathematically
identical at high prey densities.

\begin{center}\includegraphics[width=0.6\linewidth]{Statistical Challenge} \end{center}
\end{frame}

\begin{frame}{Study Background}
\phantomsection\label{study-background}
\textbf{Before Starting Experiment}:

\begin{flushleft}\includegraphics[width=300px]{model_types} \end{flushleft}

\begin{itemize}
\item
  Select a true model, this is the distribution that will be used to
  sample the number of prey consumed.
\item
  Select true values for the parameters
\item
  Select the number of iterations (I = 25), particles (N = 500), and
  time (t = 24 hrs)
\end{itemize}

\begin{flushleft}\includegraphics[width=0.8\linewidth]{pars_table} \end{flushleft}
\end{frame}

\begin{frame}{Sequential Bayesian Framework}
\phantomsection\label{sequential-bayesian-framework}
\begin{enumerate}
[1)]
\item
  Start with a design point (experiment), which contains the conditions
  of a single experiment.
\item
  Draw parameter samples (particles) from the prior distributions
\item
  Simulate data under each model at that selected design point
\item
  Compute likelihoods for each model given the simulated data.
\item
  Update particle weights, and resample if effective sample size (ESS)
  is below the threshold N/2. ESS is a measure of of the efficiency of a
  particle set.
\item
  Update model probabilities and repeat
\end{enumerate}
\end{frame}

\begin{frame}{Sensitivity Analysis of Particle Count in SMC}
\phantomsection\label{sensitivity-analysis-of-particle-count-in-smc}
\textbf{Goal}: Determine the computational efficiency of author's
original choice of N = 500 by analyzing cost-benefit trade-off.

\begin{itemize}
\item
  Sequential Monte Carlo methods approximate the parameter posterior
  distributions using particles.
\item
  SMC requires a sufficient number of particles (N) to minimize the
  Monte Carlo Variance without causing excessive computational cost, or
  runtime.
\end{itemize}
\end{frame}

\begin{frame}{Experiment Set Up}
\phantomsection\label{experiment-set-up}
Computed marginal posterior distributions for: * Binomial and beta
binomial * Type 2 and Type 3 functional response modes.

\begin{itemize}
\item
  Kept design strategy constant at R = 0 (random design) to isolate
  effect of particle count N from the sequential design choices.
\item
  Tested four discrete particle counts:
\end{itemize}

\[N \in \{30, 100, 500 \text{ (Author's Choice)}, 1000\}\]

\begin{itemize}
\tightlist
\item
  Due to high computational costs, we only performed a Single-Run
  Sensitivity Analysis. We will be looking at runtime vs.~posterior
  smoothness results.
\end{itemize}
\end{frame}

\begin{frame}{Qualitative Results: N = 30 versus N = 500}
\phantomsection\label{qualitative-results-n-30-versus-n-500}
\begin{columns}[T]
\begin{column}{0.48\linewidth}
N = 30
\end{column}

\begin{column}{0.48\linewidth}
N = 500
\end{column}

Highly unreliable estimates for N = 30 and multi peaked

\begin{frame}{Qualitative Results: N = 100 versus N = 500}
\phantomsection\label{qualitative-results-n-100-versus-n-500}
\begin{column}{0.48\linewidth}
N = 100
\end{column}

\begin{column}{0.48\linewidth}
N = 500
\end{column}
\end{frame}
\end{columns}

Improved smoothness for N = 100, still small bumps in the tails
\end{frame}

\begin{frame}{Qualitative Results: N = 1000 versus N = 500}
\phantomsection\label{qualitative-results-n-1000-versus-n-500}
\begin{column}{0.48\linewidth}
N = 1000
\end{column}

\begin{column}{0.48\linewidth}
N = 500
\end{column}
\end{frame}

N = 1000 has smoothest distributions, due to the Law of Large Numbers
dictating that as N increases, the Monte Carlo variance decreases. We
can be confident that N = 1000 and N = 500 are the statistically stable
options, but have to take into account computation cost.
\end{frame}

\begin{frame}{Comparing Run Time}
\phantomsection\label{comparing-run-time}
\begin{center}\includegraphics[width=0.5\linewidth]{N size and computational time} \end{center}

\begin{center}\includegraphics[width=0.5\linewidth]{percent increase in wait time} \end{center}
\end{frame}

\begin{frame}{\textbf{Overall}}
\phantomsection\label{overall}
For maximal statistical reliability, N = 1000 is desired. But due to
computational constraints as well as changes in design (where R is not 0
or random), the author's choice of N = 500 seems like a pragmatic
computational compromise.
\end{frame}

\begin{frame}{Setup For Determining Best Experimental Design}
\phantomsection\label{setup-for-determining-best-experimental-design}
\textbf{Goal}: Explore the best method for determining experimental
design points that maximize posterior probability for true model and
accurately estimates the true parameters.

\begin{itemize}
\item
  Due to computational complexity with method 1, the number of
  iterations and particles were scaled down to 10 and 200, respectively.
\item
  All 3 methods were tested, with the standard true parameters.
\item
  For this experiment, Model 1 was selected as the true model
\end{itemize}
\end{frame}

\begin{frame}{Results - Visual}
\phantomsection\label{results---visual}
Marginal Posterior Distributions for Beta Binomial Type 2 Functional
Response

\begin{center}\includegraphics[width=1\linewidth]{plot_results} \end{center}
\end{frame}

\begin{frame}{Results - Quantitative}
\phantomsection\label{results---quantitative}
\begin{center}\includegraphics[width=1\linewidth]{results_sum} \end{center}

\textbf{Overall}:

\begin{itemize}
\item
  All methods appeared to have relatively similar accuracy when
  estimating the true parameters
\item
  Method 1 best recovers the true model, Method 0 is uninformative, and
  Method 6 performed poorly
\end{itemize}
\end{frame}

\begin{frame}{Why Move Steps are Used}
\phantomsection\label{why-move-steps-are-used}
\begin{itemize}
\item
  Sequential Monte Carlo involves taking weighted samples (particles)
  and iteratively changing them to more closely match a target
  distribution.
\item
  To get a new posterior distribution for each iteration of Sequential
  Monte Carlo, each particle is re-weighted. However, these weights are
  often skewed, and the effective sample size is reduced.
\item
  When the effective sample size is below a threshold, it is best to
  re-sample and conduct a move step to diversify the particles, since
  duplicates often occur.
\item
  Moffat et al.~(2020) uses one move step, but outlines that is may be
  too few to diversify the particle set. Thus, we explore using two.
\end{itemize}
\end{frame}

\begin{frame}{Two-Step Move Step}
\phantomsection\label{two-step-move-step}
\begin{itemize}
\item
  The appropriate amount of times to conduct a move step for each
  particle is outlined as: \[R_m \ge \frac{\log{c}}{\log{(1-p)}}\] where
  \(c\) is a pre-selected probability for the particle to move and \(p\)
  is acceptance probability.
\item
  We know that having two steps increases the uniqueness of the particle
  set.
\item
  We know that the probability is greater for each particle to move with
  two rounds
\item
  We aim to find whether diversifying the particles will improve the
  random models' posterior distributions.
\end{itemize}
\end{frame}

\begin{frame}{One Step with Type II as True Model}
\phantomsection\label{one-step-with-type-ii-as-true-model}
\begin{itemize}
\tightlist
\item
  The model of the Type II response (true) is on the left and the model
  of the Type III (false) response is on the right.
\end{itemize}
\end{frame}

\begin{frame}{One Step with Type III as True Model}
\phantomsection\label{one-step-with-type-iii-as-true-model}
\begin{itemize}
\tightlist
\item
  The model of the Type II response (false) is on the left and the model
  of the Type III (true) response is on the right.
\end{itemize}
\end{frame}

\begin{frame}{Two Steps with Type II as True Model}
\phantomsection\label{two-steps-with-type-ii-as-true-model}
\begin{itemize}
\tightlist
\item
  The model of the Type II response (true) is on the left and the model
  of the Type III (false) response is on the right.
\end{itemize}
\end{frame}

\begin{frame}{Two Steps with Type III as True Model}
\phantomsection\label{two-steps-with-type-iii-as-true-model}
\begin{itemize}
\tightlist
\item
  The model of the Type II response (false) is on the left and the model
  of the Type III (true) response is on the right.
\end{itemize}
\end{frame}

\begin{frame}{Take Aways}
\phantomsection\label{take-aways}
\begin{itemize}
\item
  It appears that overall, having two steps helps the distributions
  match their true posteriors more accurately.
\item
  Moffat et al., (2020) explores much better approaches, but these are
  very expensive. This approach likely increases significantly.
\end{itemize}
\end{frame}

\end{document}
